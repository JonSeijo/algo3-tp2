% !TEX root = ./informe.tex

\section{Problema 3}
\subsection{Explicación}

En una provincia, hay ciudades conectadas por rutas bidireccionales. No todas estan conectadas. Se quiere que exista una única forma de llegar de una ciudad a cualquier otra. Para logar esto, se pueden \textbf{destruir} rutas existentes o \textbf{construir} nuevas rutas. La construcción y destrucción de cada ruta tiene un costo asociado, por lo que se quiere además minimizar el costo. \\

Para resolver el problema podemos pensarlo como un problema de grafos. La provincia es un \textit{grafo}, donde cada ciudad es un \textit{nodo} y las rutas son \textit{aristas}.  \\

Si consideramos al grafo como el completo de \textit{n} nodos (donde \textit{n} es la cantidad de ciudades). Que exista una y solo una ruta para llegar de una ciudad a cualquier otra, significa que tenemos que lograr que las rutas existentes formen un árbol (que incluya a todos los nodos).  \\

Como podemos construir rutas nuevas y destruit las existentes, podríamos en principio quedarnos con cualquier árbol generador del grafo completo de \textit{n} nodos. Esto hace que las rutas elegidas cumplan la condición de conexiones. Restaría considerar que las rutas elegidas tienen además que tener el mínimo costo posible. \\

Las observaciones claves son las siguientes: \\
- Si se tiene que elegir entre construir dos rutas que no existen, lo mejor es construir la mas barata. \\
- Si se tiene que elegir entre destruir dos ya existentes, es mejor quedarse con la mas cara de destruir. \\
- Si se tiene que elegir entre mantener una ruta existente o destruirla y construir otra, es mas barato manterla. Mantener una ruta cuesta 0, mientras que destruirla y construir otra tiene costo. \\

Teniendo esto en cuenta, podemos armar nuestro grafo de la siguiente manera: \\

Tomamos un grafo completo de \textit{n} nodos. Las rutas que \textbf{no} existian las colocamos con su costo de construcción. Las rutas que \textbf{sí} existían las colocammos con el peso \textbf{negativo} de su destrucción. ¿Por qué el negativo? Al tratar de elegir los menores costos, se prioriza elegir una ya construida a una que no lo está, y entre dos construidas prioriza aquella que cuesta mas destruir. \\

La solución final es el Arbol Generador Mínimo de este grafo. \\



\subsection{Correctitud}



\subsection{Pseudocódigo}
\todo[inline]{Falta agregar comentarios sobre las complejidades}

Vamos a utilizar como entrada en nuestro algoritmo a las siguientes variables:
\begin{itemize}
	\item $n$: La cantidad de ciudades
	\item $existe$: El grafo de entrada representado con matriz de adyacencia.
	\item $costo$: La matriz con los costos de construcci\'on o destrucci\'on.
\end{itemize}

\begin{algorithm}
\label{ej3}         % and a label for \ref{} commands later in the document
\begin{algorithmic}
\Function{$resolver$}{}
	\State $costoInicialDestruirTodo \gets negativizarCostoConstruidas()$
	\State $arbol \gets primNaive()$
	\State \Return \Call{$obtenerCostoTotal$}{$arbol, costoInicialDestruirTodo$}
\EndFunction

\Function{$negativizarCostoConstruidas$}{}
	\State $costoInicialDestruirTodo \gets 0$
	\For{$i \in [0..n)$}
		\For{$j \in [1..n]$}
			\If{$existe[i][j]$}
				\State $costoInicialDestruirTodo \gets costoInicialDestruirTodo + costo[i][j]$
				\State $costo[i][j] \gets -costo[i][j]$
				\State $costo[j][i]$
			\EndIf
		\EndFor
	\EndFor
	\State \Return $costoInicialDestruirTodo$
\EndFunction

\Function{$primNaive$}{}
	\State $visitado \gets Bool[n+1]$
	\State $\forall i visitado[i] \gets false$
	\State $dist \gets Int[n+1]$
	\State $\forall i dist[i] \gets INFINITO$	
	\State $padre \gets Int[n+1]$
	\State $\forall i padre[i] \gets -1$
	\State $s \gets 1$
	\For{$w \in [1..n]$}
		\If{$s \neq w$}
			\State $dist[w] \gets costo[s][w]$
			\State $padre[w] \gets s$
		\EndIf
	\EndFor
	\State $dist[s] \gets 0$
	\State $visitados[s] \gets true$
	\For{$repes \in [1..n-1]$}
		\State $v \gets -1$
		\State $minDist \gets INFINITO$
		\For{$u \in [1..n]$}
			\If{$\neg visitados[u] \land dist[u] < minDist$}
				\State $v \gets u$
				\State $minDist \gets dist[u]$
			\EndIf
		\EndFor
		\State $visistado[u] \gets true$
		
		\For{$w \in [1..n]$}
			\If{$(\neg visitado[w]) \land (costo[v][w] < dist[w])$}
				\State $dist[w] \gets costo[v][w]$			
				\State $padre[w] \gets v$
			\EndIf
		\EndFor
	\EndFor
	\State \Return $padre$
\EndFunction

\Function{$obtenerCostoTotal$}{$arbol: Int[], costoInicialDestruirTodo: Int$}
	\State $costoTotal \gets costoInicialDestruirTodo$
	\For{$i \in [2..n]$}
		\State $j \gets arbol[i]$
		\State $costoTotal \gets costoTotal + costo[i][j]$
	\EndFor
	\State \Return $costoTotal$
\EndFunction
\end{algorithmic}
\end{algorithm}

\newpage
\subsection{Complejidad}

$$O(n^2)$$ usando Prim naive


\subsection{Experimentos}

ideas:\\
0 rutas existentes\\
m rutas existentes\\
random rutas existentes\\

En general todo deberia dar los mismos tiempos, porque siempre construyo el arbol completo y aplico prim al completo