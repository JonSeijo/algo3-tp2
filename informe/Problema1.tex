% !TEX root = ./informe.tex

\section{Problema 1}


\subsection{Introduccion}
\subsection{Correctitud}
Queremos representar cada ciudad como un vértice y cada ruta como una arista con el objetivo de calcular el camino mínimo entre el origen y el destino mediante algún algoritmo conocido y probado para este propósito. Llamaremos a este grafo $G_0$\\

El principal inconveniente de este planteo es el modelado de las rutas premium, las cuales (y según el valor de $k$) pueden restringir las aristas disponibles para el cálculo del camino mínimo a medida que se recorran vértices utilizando estas rutas.\\

Es importante notar que existen diferentes estados para cada vértice, a modo de ejemplo, para $k=1$ y ningún camino premium recorrido, los vértices pueden utilizar cualquier arista, por lo que si $v_i$ es adyacente a $v_j$ para el input original, entonces seguirá siéndolo. Este no es el caso si se ha recorrido un camino premium, en este caso si la arista $(v_i, v_j)$ fuera premium entonces $v_i$ ahora no es adyacente a $v_j$. En conclusión, los estados de los vértices dependen de la cantidad de aristas premium recorridas.\\

Siendo $K$ la máxima cantidad de caminos premium a recorrer, cada vértice puede tener hasta $K$ estados diferentes. Se resuelve modelar un digrafo $G_1$ y representar a cada estado como un vértice $v_i^k$ en el cual sus aristas responden a las siguientes reglas:
\begin{itemize}
	\item Si $(v_i,v_j) \in E_{G_0} \land (v_i,v_j) \notin premium \Rightarrow (v_i^k,v_j^k) \in E_{G_1} \land (v_j^k,v_i^k) \in E_{G_1}$
	\item Si $(v_i,v_j) \in E_{G_0} \land (v_i,v_j) \in premium \land k < K \Rightarrow (v_i^k,v_j^{k+1}) \in E_{G_1}$
\end{itemize}
En este modelo no es posible recorrer más de $K$ aristas premium, dado que cada vez que se recorre cualquier premium se pasa de un vértice $k$ a uno $k+1$ (exceptuando $k=K$ en el cual no existe vértice premium para ningún $v^k$) y sólo existen $K$ estados disponibles.\\\\
Además, al tratarse de un simple digrafo con aristas positivas, se puede calcular el camino mínimo entre el origen y todos los demás vértices utilizando el algoritmo de Dijkstra, para luego obtener $min(v_{destino}^k \forall k)$, la distancia mínima entre origen y destino pasando por a lo sumo $k$ aristas premium. Siendo el origen en $v_i^0$ para algún $i$ especificado en la entrada, esto es así porque el vértice de origen no recorrió ninguna arista, en particular ninguna arista premium, por lo que pertenece al estado $k=0$, además los vértices $\{v_j^0,v_j^1,...,v_j^k\}$ representan al vértice destino para un $j$ especificado en la entrada, siendo $k$ la cantidad de aristas premium recorridas, por lo que al aplicar dijkstra sobre $G_1$ empezando por el origen, tendremos todas las distancias de este hacia los vértices destino (de distintos $k$), entre los cuales hay que elegir el que tenga una distancia mínima. \\
\subsection{Pseudocodigo}
\subsection{Complejidad}
\subsection{Experimentos}