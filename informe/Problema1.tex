% !TEX root = ./informe.tex

\section{Problema 1}

\subsection{Introduccion}

\subsection{Correctitud}
Queremos representar cada ciudad como un vértice y cada ruta como una arista con el objetivo de calcular el camino mínimo entre el origen y el destino mediante algún algoritmo conocido y probado para este propósito. Llamaremos a este grafo $G_0$\\

El principal inconveniente de este planteo es el modelado de las rutas premium, las cuales (y según el valor de $k$) pueden restringir las aristas disponibles para el cálculo del camino mínimo a medida que se recorran vértices utilizando estas rutas.\\

Es importante notar que existen diferentes estados para cada vértice, a modo de ejemplo, para $k=1$ y ningún camino premium recorrido, los vértices pueden utilizar cualquier arista, por lo que si $v_i$ es adyacente a $v_j$ para el input original, entonces seguirá siéndolo. Este no es el caso si se ha recorrido un camino premium, en este caso si la arista $(v_i, v_j)$ fuera premium entonces $v_i$ ahora no es adyacente a $v_j$. En conclusión, los estados de los vértices dependen de la cantidad de aristas premium recorridas.\\

Siendo $K$ la máxima cantidad de caminos premium a recorrer, cada vértice puede tener hasta $K$ estados diferentes. Se resuelve modelar un digrafo $G_1$ y representar a cada estado como un vértice $v_i^k$ en el cual sus aristas responden a las siguientes reglas:
\begin{itemize}
	\item Si $(v_i,v_j) \in E_{G_0} \land (v_i,v_j) \notin premium \Rightarrow (v_i^k,v_j^k) \in E_{G_1} \land (v_j^k,v_i^k) \in E_{G_1}$
	\item Si $(v_i,v_j) \in E_{G_0} \land (v_i,v_j) \in premium \land k < K \Rightarrow (v_i^k,v_j^{k+1}) \in E_{G_1}$
\end{itemize}
En este modelo no es posible recorrer más de $K$ aristas premium, dado que cada vez que se recorre cualquier premium se pasa de un vértice $k$ a uno $k+1$ (exceptuando $k=K$ en el cual no existe vértice premium para ningún $v^k$) y sólo existen $K$ estados disponibles.\\\\
Además, al tratarse de un simple digrafo con aristas positivas, se puede calcular el camino mínimo entre el origen y todos los demás vértices utilizando el algoritmo de Dijkstra, para luego obtener $min(v_{destino}^k \forall k)$, la distancia mínima entre origen y destino pasando por a lo sumo $k$ aristas premium. Siendo el origen en $v_i^0$ para algún $i$ especificado en la entrada, esto es así porque el vértice de origen no recorrió ninguna arista, en particular ninguna arista premium, por lo que pertenece al estado $k=0$, además los vértices $\{v_j^0,v_j^1,...,v_j^k\}$ representan al vértice destino para un $j$ especificado en la entrada, siendo $k$ la cantidad de aristas premium recorridas, por lo que al aplicar dijkstra sobre $G_1$ empezando por el origen, tendremos todas las distancias de este hacia los vértices destino (de distintos $k$), entre los cuales hay que elegir el que tenga una distancia mínima. \\

\subsection{Pseudocodigo}

Vamos a utilizar como entrada en nuestro algoritmo las siguientes variables:
\begin{itemize}
	\item $n$: La cantidad de ciudades
	\item $m$: La cantidad de aristas
	\item $k$: La máxima cantidad de rutas premium posibles de recorrer
	\item $origen$: La ciudad origen
	\item $destino$: La ciudad destino
	\item $matrizAdy$: El grafo de entrada representado con matriz de adyacencia.
	\item $esPremium$: La matriz de booleanos que indica si una ruta es premium o no
\end{itemize}

\begin{algorithm}[H]
% \label{ej3}         % and a label for \ref{} commands later in the document
\begin{algorithmic}
\Function{Resolver}{}    \Comment{$\mathbf{\mathcal{O}(n^2*k^2)}$}
	\State $visitado \gets$ InitMatrizEnFalso($n*(k+1)+1$)    \Comment{$\mathcal{O}(n^2*k^2)$}
	\State $dist \gets$ InitMatrizEnInfinito($n*(k+1)+1$)    \Comment{$\mathcal{O}(n^2*k^2)$} \\
	
	\State $s = origen$    \Comment{$\mathcal{O}(1)$}
	\For{$w \in [1..n]$}    \Comment{$\mathcal{O}(n)$}
		\If{$matrizAdy[s][w] \not = 0$}    \Comment{$\mathcal{O}(1)$}
			\If{$esPremium[s][w]$}    \Comment{$\mathcal{O}(1)$}
				\If{$k > 0$}    \Comment{$\mathcal{O}(1)$}
					\State $dist[w+n] \gets matrizAdy[s][w]$    \Comment{$\mathcal{O}(1)$}
				\EndIf
			\Else
				\State $dist[w] \gets matrizAdy[s][w]$    \Comment{$\mathcal{O}(1)$}
			\EndIf
		\EndIf
	\EndFor \\
	
	\State $dist[s] \gets 0$    \Comment{$\mathcal{O}(1)$}
	\State $visitado[s] \gets True$    \Comment{$\mathcal{O}(1)$} \\
	
	\State $finalizado \gets False$    \Comment{$\mathcal{O}(1)$}
	
	\While{$\neg finalizado$}    \Comment{$\mathcal{O}(n*k)$\hyperref[whilejust]{$^1$}}
		\State $v \gets -1$    \Comment{$\mathcal{O}(1)$}
		\State $minDist = \infty$    \Comment{$\mathcal{O}(1)$}
		\For{$u \in [1..n*(k+1)]$}    \Comment{$\mathcal{O}(n*k)$}
			\If{$\neg visitado[u] \land dist[u] < minDist$}    \Comment{$\mathcal{O}(1)$}
				\State $v \gets u$    \Comment{$\mathcal{O}(1)$}
				\State $minDist \gets dist[u]$    \Comment{$\mathcal{O}(1)$}
			\EndIf
		\EndFor
	
		\If{$minDist = \infty$}    \Comment{$\mathcal{O}(1)$}
			\State $finalizado \gets True$    \Comment{$\mathcal{O}(1)$}
			\State \textbf{continue} \Comment{$\mathcal{O}(1)$}
		\EndIf \\
	
		\State $visitado[v] \gets True$    \Comment{$\mathcal{O}(1)$}
		\State $nivel \gets (v-1)/n$    \Comment{$\mathcal{O}(1)$}
		\State $vOriginal \gets v - n * nivel$     \Comment{$\mathcal{O}(1)$} \\
	
		\For{$w \in [1..n]$}    \Comment{$\mathcal{O}(n)$}
			\If{$matrizAdy[vOriginal][w] \not = 0$}    \Comment{$\mathcal{O}(1)$}
				\If{$esPremium[vOriginal][w]$}    \Comment{$\mathcal{O}(1)$}
					\If{$nivel = k$}    \Comment{$\mathcal{O}(1)$}
						\textbf{continue}    \Comment{$\mathcal{O}(1)$}
					\EndIf
					\State $vecinoPremium \gets w + (nivel+1) * n$    \Comment{$\mathcal{O}(1)$}
					\If {$\neg visitado[vecinoPremium]$}    \Comment{$\mathcal{O}(1)$}
						\If{$dist[vecinoPremium] > dist[v] + matrizAdy[vOriginal][w]$}    \Comment{$\mathcal{O}(1)$}
							\State $dist[vecinoPremium] = dist[v] + matrizAdy[vOriginal][w]$    \Comment{$\mathcal{O}(1)$}
						\EndIf
					\EndIf
				\Else
					\State $vecinoComun \gets w + nivel*n$    \Comment{$\mathcal{O}(1)$}
					\If {$\neg visitado[vecinoComun]$}    \Comment{$\mathcal{O}(1)$}
						\If{$dist[vecinoComun] > dist[v] + matrizAdy[vOriginal][w]$}    \Comment{$\mathcal{O}(1)$}
							\State $dist[vecinoComun] = dist[v] + matrizAdy[vOriginal][w]$    \Comment{$\mathcal{O}(1)$}
						\EndIf
					\EndIf				
				\EndIf
			\EndIf
		\EndFor
	\EndWhile \\
	
	\State $distanciaMinima \gets \infty$    \Comment{$\mathcal{O}(1)$}
	\For{$i \in [destino, destino + n, destino + 2n \, .. \, n*(k+1)]$}    \Comment{$\mathcal{O}(k)$}
		\If{$dist[i] < distanciaMinima$}    \Comment{$\mathcal{O}(1)$}
			\State $distanciaMinima \gets dist[i]$    \Comment{$\mathcal{O}(1)$}
		\EndIf
	\EndFor \\
	
	\If{$distanciaMinima = \infty$}    \Comment{$\mathcal{O}(1)$}
		\State \Return $-1$    \Comment{$\mathcal{O}(1)$}
	\Else
		\State \Return $distanciaMinima$    \Comment{$\mathcal{O}(1)$}
	\EndIf
\EndFunction

\end{algorithmic}
\end{algorithm}

\todo[inline]{En el último for hago el 'step = n' de una forma rara, ideas bienvenidas}

\subsection{Complejidad}

\label{whilejust} $^1$ Pinpunpan 

\subsection{Experimentos}